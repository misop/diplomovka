\chapter{Conclusion}

We have developed an interactive application in which artist can create complex skeletal structures.
The generated skeletons are automatically converted to base manifold meshes.
The procedure is fully automatic and no user interaction is needed.
Our algorithm also improves on already existing algorithms.
Or algorithm can handle base meshes from linear skeletons which do not have branch nodes.
Skeletons, which root is not a branch node, can also be generated naturally with our algorithm.
Explicitly defined cyclic skeletons of arbitrary topology can be set as input to our algorithm.
The algorithm is capable of generating a base mesh with genus corresponding to the input skeleton.
To accelerate the computation of the algorithm we have implemented one step of the algorithm in GPU, thereby saving CPU processing time.
We have also generalized spherical nodes of the input skeleton to ellipsoids.
The computation of ellipsoid nodes is also accelerated on the GPU.
Lastly we are using tessellation shaders to subdivide the generated base mesh and improve its visual quality.

The generated base meshes have good edge flow and can be animated with standard animation techniques.
Base meshes consists mainly from quadrilateral faces.
Triangular faces are present only at leaf nodes.
Computation of a base mesh is very fast as even complex meshes can be computed in milliseconds.
Our technique is fully automatic and does not require any additional input from the user during generation of base meshes.

\section{Future work}

\paragraph{Parallel computation}
Every step of the algorithm has just one pre-condition.
When we are processing a node the only pre-condition is that the nodes parent was already processed.
We could take advantage of this fact and accelerate base mesh generation even further.
At each branch node we can start a parallel computation of its child nodes.
With faster base mesh generation we are able to use base meshes as placeholder in complex applications.
Fully detailed animated models are expensive and time consuming.
Our method could generate base meshes on the fly directly during animation.
This way the animation can be produced meanwhile artist create more sophisticated and detailed models.

\paragraph{Procedural mesh generation}
Skeletal structures corresponding to plants can be generated using L-Systems \cite{lplants}.
Such skeletal structures could be send directly into our application and their corresponding base meshes could be recovered using our algorithm.
Procedurally generated displacement maps could be used to add finer details to the generated base meshes of plants.
Entire bioms of procedurally generated plants could be generated on the fly.