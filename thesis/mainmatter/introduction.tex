\chapter*{Introduction}
\addcontentsline{toc}{chapter}{Introduction}

Skeletal structures are often used in computer graphics to represent basic topology of a model.
This representation allows artists to conveniently animate articulated models, by only manipulating key points represented as nodes in skeletons.
Skeletons corresponding to a model are often provided by an artist or extracted directly from a model \cite{laplac}.
Since skeletal structures carry an information about the topology of  a model, we could apply a reverse process to skeleton extraction and recover the base mesh represented by a skeleton.

Such base meshes, generated directly from skeletal structures, could be used to ease the modelling of base models of articulated characters.
An artist would only design the skeleton of the model and the base mesh would be generated automatically.

This technique can also be used to procedurally generate articulated models.
A skeleton can be either procedurally generated or manually entered.
A base mesh generated from a supplied skeleton can be augmented, with procedurally generated displacement maps, to generate a complex model.

In this thesis we will start with presenting the related work in the field of generating base meshes from skeletal structures.
We will analyse each proposed solution and pick the one best suited for our needs as base of our own algorithm.
Next, we will describe the functionality of our own solution, the implementation and proposed extensions.
In the following chapter, we will present implementation details of our system, used programming language, libraries and design patterns.
In the last two chapters, we will present results of our work and summarize our goals in conclusion.